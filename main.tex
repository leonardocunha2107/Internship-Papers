\documentclass{article}
\usepackage[a4,  margin=4 cm]{geometry}


% \usepackage{natbib}
%\usepackage{ICLR/natbib}
\usepackage{ICLR/iclr2022_conference}

\usepackage[utf8]{inputenc}

\usepackage{comment}
\usepackage{csquotes}


%%proof at end


\usepackage[ruled,vlined]{algorithm2e}
\usepackage[english]{babel}
\usepackage{url}
\usepackage{mathtools}
\usepackage{enumitem}
\usepackage{dsfont}
\usepackage{bbm}
\usepackage{float}
\usepackage{stmaryrd}
\usepackage[utf8]{inputenc}
\usepackage[T1]{fontenc}
\usepackage{placeins}
\usepackage{amsmath, amsfonts, amssymb}
\usepackage{graphicx}
\usepackage{caption}
\usepackage[nottoc]{tocbibind}
\usepackage[ruled,vlined]{algorithm2e}
\newcommand\tab[1][0.5cm]{\hspace*{#1}}
\newcommand\reals{\mathbb{R}}
\usepackage{amsthm}
\newcommand\tp{\Tilde{P}}
\newcommand\JP{P^{\alpha,\beta}}

\newcommand\tmu{\Tilde{\mu}}
\DeclareMathOperator{\supp}{supp}

\newtheorem{theorem}{Theorem}
\newtheorem{define}{Definition}
\newcommand\jp{p_t^{\alpha,\beta}}
\newtheorem{lemma}[theorem]{Lemma}
%\newtheorem{assumption}[theorem]{Assumption}
\newtheorem{corolary}[theorem]{Corollary}
%\newtheorem{example}[theorem]{Example}

\newtheorem{remark}[theorem]{Remark}
\usepackage[utf8]{inputenc} % allow utf-8 input
\usepackage[T1]{fontenc}    % use 8-bit T1 fonts
%\usepackage{hyperref}       % hyperlinks
\usepackage{url}            % simple URL typesetting
\usepackage{booktabs}       % professional-quality tables
\usepackage{amsfonts}       % blackboard math symbols
\usepackage{nicefrac}       % compact symbols for 1/2, etc.
\usepackage{microtype}      % microtypography

\usepackage{appendix}


\usepackage{empheq}
\usepackage{xcolor, color, colortbl}
\usepackage{framed}
\usepackage{pifont}
\newcommand{\cmark}{\ding{51}}%
\newcommand{\xmark}{\ding{55}}%
\usepackage{enumitem}
% For figures
\usepackage{graphicx} % more modern
%\usepackage{epsfig} % less modern

% to avoid too much space after the algorithm
% \setlength{\textfloatsep}{10pt}% Remove \textfloatsep

\usepackage{nicefrac}
\usepackage{booktabs}
\usepackage{multirow}
\usepackage{rotating}
\usepackage{nccmath}

\usepackage[tikz]{bclogo}
\usepackage{wasysym}

%\SetAlFnt{\normalsize}

\definecolor{mydarkblue}{rgb}{0,0.08,0.45}
\usepackage[
    colorlinks=true,
    linkcolor=mydarkblue,
    citecolor=mydarkblue,
    filecolor=mydarkblue,
    urlcolor=mydarkblue,
    pdfview=FitH]{hyperref}
% smaller url fonts
\usepackage{url}
\usepackage{relsize}
\renewcommand*{\UrlFont}{\ttfamily\smaller\relax}

\newcommand{\blue}{\color{blue}}
\def\xx{{\boldsymbol x}}
\def\yy{{\boldsymbol y}}
\def\zz{{\boldsymbol z}}
\def\qq{{\boldsymbol q}}
\def\dd{{\boldsymbol d}}
\def\XX{{\boldsymbol X}}
\def\YY{{\boldsymbol Y}}
\def\ZZ{{\boldsymbol Z}}
\def\aa{{\boldsymbol a}}
\def\bb{{\boldsymbol b}}
\def\rr{{\boldsymbol r}}
\def\cc{{\boldsymbol c}}
\def\qq{{\boldsymbol q}}
\def\WW{{\boldsymbol W}}
\def\KK{{\boldsymbol K}}
\def\II{{\boldsymbol I}}
\def\yy{{\boldsymbol y}}
\def\vv{{\boldsymbol v}}
\def\uu{{\boldsymbol u}}
\def\ww{{\boldsymbol w}}
\def\zz{{\boldsymbol z}}
\def\SS{{\boldsymbol S}}
\def\BB{{\boldsymbol B}}
\def\AA{{\boldsymbol A}}
\def\CC{{\boldsymbol C}}
\def\MM{{\boldsymbol M}}
\def\DD{{\boldsymbol D}}
\def\PP{{\boldsymbol P}}
\def\TT{{\boldsymbol T}}
\def\VV{{\boldsymbol V}}
\def\bphi{{\boldsymbol \phi}}
\def\QQ{{\boldsymbol Q}}
\def\UU{{\boldsymbol U}}
\def\HH{{\boldsymbol H}}
\def\balpha{{\boldsymbol \alpha}}
\def\bpsi{{\boldsymbol \psi}}
\def\LLambda{{\boldsymbol \Lambda}}
\def\eeps{{\boldsymbol \varepsilon}}
\def\dif{\mathop{}\!\mathrm{d}}
\def\Proba{\mathbb{P}}
\def\MP{\mu_{\mathrm{MP}}}

\def\RR{{\mathbb R}}
\def\EE{{\mathbb{E}}}
\newcommand{\Econd}{\mathbf{E}}
\renewcommand{\gg}{\boldsymbol{g}}
\renewcommand{\vec}{\mathbf{vec}}
\DeclareMathOperator{\prox}{\mathbf{prox}}
\DeclareMathOperator{\tr}{tr}
\def\defas{\stackrel{\text{def}}{=}}
\DeclareMathOperator*{\dom}{dom}
% \DeclareMathOperator*{\supp}{supp}
\DeclareMathOperator*{\Fix}{Fix}
\DeclareMathOperator*{\Var}{Var}
\DeclareMathOperator*{\Span}{\mathbf{span}}
\DeclareMathOperator*{\Deg}{{deg}}
\DeclareMathOperator*{\Trace}{{tr}}

% \newcommand{\xdownarrow}[1]{%
%   {\left\downarrow\vbox to #1{}\right.\kern-\nulldelimiterspace}
% }


\newcommand*\mybluebox[1]{\colorbox{myblue}{\hspace{1em}#1\hspace{1em}}}

\DeclareMathOperator*{\argmin}{{arg\,min}}
\DeclareMathOperator*{\minimize}{minimize}
\DeclareMathOperator*{\diag}{\mathbf{diag}}

\definecolor{myblue}{HTML}{D2E4FC}
\definecolor{Gray}{gray}{0.92}
    
\newtheorem{definition}{Definition}
\newtheorem{example}{Example}
% \newtheorem{remark}{Remark}
\newtheorem{assumption}{Assumption}
% \newtheorem{lemma}{Lemma}
% \newtheorem{proof}{Proof}
\newtheorem{proposition}{Proposition}
% \newtheorem{theorem}{Theorem}


\usepackage{thmtools}
\usepackage{thm-restate}

\usepackage{wrapfig}

\declaretheorem[name=Theorem,numberwithin=section]{thm}
\declaretheorem[name=Proposition,numberwithin=section]{prop}

\DeclareDocumentCommand{\Prto} {o} {
  \IfNoValueTF {#1}
  {\overset{\Pr}{\longrightarrow}}
  { \xrightarrow[ #1 \to \infty]{\Pr }}
}


\title{Average Case Universality in the Convex setting}


\author{Leonardo Cunha
\and Gauthier Gidel \\
\and Fabian Pedregosa \\
\and Courtney Paquette \\
\and Damien Scieur}


\date{August 2021}

\begin{document}
\maketitle
\begin{abstract}
    Recent works \cite{pedregosa2020acceleration,paquette2020halting,scieur2020universal} have studied the convergence properties of first order optimization methods on distributions of quadratic problems. 
    The average case framework allows for a more fine grained and representative analysis of convergence in exchange of a more precise hypothesis over the data, namely the expected spectral distribution. \\
    In this work we introduce a middle ground between the coarseness of the  worst case analysis and the unrealistic hypothesis of an exact spectrum by showing that, in the asymptotic sense, a problem's complexity is driven to the concentration of near the edges it's eigenspectra. \\
    We introduce the Generalized Chebyshev method, which is   asymptotically optimal under an hypothesis on this concentration, and globally optimal when the spectrum is a beta weight.\\
    We compare it`s performance to the Nesterov Method \cite{nesterov2003introductory}, and Gradient descent, showing that Nesterov is robustly close to the asymptotic optimum in practical scenarios.
\end{abstract}
\section{Introduction}
The analysis of the average complexity of algorithms is an old feature of computer science. Notably, the quicksort algorithm presents worst-case complexity of $O(n^2)$, but average case complexity of $O(n\log n)$ and is  empirically faster than HeapSort or MergeSort. Average case complexity also drives much of the decisions made in cryptography \cite{bogdanov2006average}\\
The average case analysis of optimization algorithm has stayed an unexplored problem for long because of the ill-defined notion of a distribution over the optimization problems. Recently though, \cite{pedregosa2020acceleration} has derived a framework to systemically evaluate the complexity of first order methods on distributions of quadratic minimization problems by tying the average of the residuals to the \textit{expected spectral distribution} of the problem, which is a well studied object on Random Matrix Theory. \\
\cite{paquette2020halting} has expanded on this work by introducing a new generative model for the problems and deriving the average complexity of the Nesterov Accelerated Method on a particular distribution and showing the strong concentration of the metrics around the infinite-dimensional limit \\
\cite{scieur2020universal} has shown that for a strongly convex problem with eigenvalues supported on a contiguous interval, the optimal average case complexity is asymptotically equal to the one given by the Polyak Heavy Ball method in the worst-case.  

%%There are more works on the average case I did not introduce , such as those done on the stochastic setting and on CG. They are not directly relevant to our work, IDK if I should introduce them aswell 

\paragraph{}
In the non-strongly convex case, optimization drastically slows down, as gradient descent presents worst case convergence in $\Theta(\frac{1}{n})$ and Nesterov is $\Theta(\frac{1}{n^2})$, which matches a lower bound on worst case convergence up to a constant factor \cite{nesterov2003introductory}. \\
We argue that, differently from the strongly convex case where we asymptotically identify worst and average case, in the simply convex scenario is very pessimistic. Especially in high dimensions  the problem has enough degrees of freedom to be highly adversarial. \\
These differences become important because, as we are dealing with the sublinear convergence of large scale algorithms,  accelerated sublinear rates may be the difference between problems that are computationally feasible and those who are not  \\
A drawback on average case analysis is that it relies, \textit{à priori}, on a much stronger hypothesis, the shape of the \textit{expected spectral distribution} as we'll see, when compared to the worst case analysis, that relies only on the values of the edges of this distribution. \\
We show that the main aspects of the convergence, which from an optimization point view are the asymptotic rates, are determined by a low dimensional characterization of the distribution. Parametrizing the distributions in terms of the \textbf{concentration} of the eigenvalues around the edges, allow us to effectively determine the precise rates for all continuous distributions supported in an interval $]0,L[$, \\
We can then compare algorithms robustly, allowing an effective choice of algorithms under a more realistic lack of information on the eigenspectra. \\
%Considering that, we were motivated to look for average case speed ups or  similar universality results on the non-strongly convex setting, which is  also a more relevant model for the optimization problems commonly found on machine learning. A positive answer would diminish the relevance of the average case analysis, as the average rates would be very similar to the worst case ones.\\
%As we found out, in this setting different distributions gave origin to different sublinear rates, indeed the worst case complexity can be arbitrarily pessimistic depending on the distribution as we'll see. Our concern for universality got changed into a concern for robustness. We can thus phrase the question this paper tries to answer as the search for the method who has "good" performance under the broadest set of hypothesis over the problem distribution. \\
\paragraph{}
In the following, we will first present the average case framework first introduced by \cite{pedregosa2020acceleration}, we'll then present some basic universality results showing that the complexity is driven by the concentration the eigenvalues near the edges of the support. \\
We will then compute the optimal rates those  of Gradient Descent and of Nesterov in terms of the concentration parameters, and we observe we can find the worst case rates as limit on the average case. \\
Finally we'll state our main results : the Nesterov accelerated method asymptotic rate is close to the optimal one for relatively high concentrations of the eigenspectra near $0$, only a $\log$ term under a standard hypothesis. This means the Nesterov method is a very robust choice on practical scenarios.


\begin{figure}[H]
    \centering
    \includegraphics[width=8 cm]{imgs/concentration.PNG}
    \caption{The average case rates for non-strongly problems is determined by the eigenvalue concentration around the edges of the support }
    \label{fig:my_label}
\end{figure}

\section{Average-Case Analysis} \label{sec:methods}


In this section we introduce the average-case analysis framework for random quadratic problems.
The main result is Theorem~\ref{thm: metrics}, which relates the expected error with other quantities that will be easier to manipulate, such as the residual polynomial. This is a convenient representation of an optimization method that will allow us in the next section to pose the problem of finding an optimal method as a best approximation problem in the space of polynomials.


Let $\HH \in \RR^{d \times d}$ be a random symmetric positive-definite matrix and $\xx^\star \in \RR^d$ a random vector. These elements determine the following (random) quadratic minimization problem
\begin{empheq}[box=\mybluebox]{equation*}\tag{OPT}\label{eq:quad_optim}
  \vphantom{\sum_0^i}\min_{\xx \in \RR^d} \Big\{ f(\xx) \defas\!\mfrac{1}{2}(\xx\!-\!\xx^\star)^\top\!\HH(\xx\!-\!\xx^\star) \Big\}\,.
\end{empheq}
Our goal is to quantify the expected error $\EE \|\xx_t - \xx^\star\|$, where $\xx_t$ is the $t$-th update of a first-order method starting from $\xx_0$ and $\EE$ is the expectation over the random $\HH, \xx_0, \xx^\star$.



\begin{remark} The expectation in the expected error ${\EE \|\xx_t - \xx^\star\|^2}$ is over the inputs and not over any randomness of the algorithm, as would be common in the stochastic literature. In this paper we will only consider deterministic algorithms.
\end{remark}


To solve \ref{eq:quad_optim}, we will consider \emph{first-order methods}. These are methods in which the sequence of iterates $\xx_t$ is in the span of previous gradients, i.e.,
\begin{equation} \label{eq:first_order_methods}
    \xx_{t+1} \in \xx_0 + \Span\{ \nabla f(\xx_0), \ldots, \nabla f(\xx_t)  \}\, .
\end{equation}
This class of algorithms includes for instance gradient descent and momentum, but not quasi-Newton methods, since the preconditioner could allow the iterates to go outside of the span. Furthermore, we will only consider \emph{oblivious} methods, that is, methods in which the coefficients of the update are known in advance and don't depend on previous updates. This leaves out some methods that are specific to quadratic problems like conjugate gradient.


\paragraph{From First-Order Method to Polynomials.}
There is an intimate link between first order methods and polynomials that simplifies the analysis on quadratic objectives. Using this link, we will be able to assign to each optimization method a polynomial that determines its convergence. The next Proposition gives a precise statement:
% The following proposition states this relationship which relates the error at iteration $t$ with the error at initialization and the residual polynomial.


\begin{restatable}{prop}{linkalgopolynomial}
    \label{prop:link_algo_polynomial} \parencite{hestenes1952methods}
    Let $\xx_t$ be generated by a first-order method. Then there exists a polynomial $P_t$ of degree $t$ such that $P_t(0) = 1$ that verifies
    \begin{equation}\label{eq:polynomial_iterates}
        \vphantom{\sum^n}\xx_{t}-\xx^\star = P_t(\HH)(\xx_0-\xx^\star)~.
    \end{equation}
\end{restatable}

\begin{remark} \label{rmk: momentum based}
If the first-order method is further a \textbf{momentum method}, i.e.:
$$
    \xx_{t+1}=\xx_t+m_t(\xx_t-\xx_{t-1})+h_t\nabla f(\xx_t)
$$.
We can determine the polynomials by the recurrence $P_0=1$ and:
    \begin{equation*}
        P_{t+1}(\lambda)=(1+m_t)P_t(\lambda)+h_t\lambda P_t(\lambda)-m_tP_{t-1}(\lambda)
    \end{equation*}
We note that while most popular F.O.M's can be posed a momentum method, Nesterov cannot.
\end{remark}

Following \textcite{fischer1996polynomial}, we will refer to this polynomial $P_t$ as the \emph{residual polynomial}.




A convenient way to collect statistics on the spectrum of a matrix is through its \emph{empirical spectral distribution}, which we now define. 


\begin{definition}
(\textbf{Weighted/Expected spectral distribution}). 
Let $\HH$ be a random matrix with eigenvalues $\{\lambda_1, \ldots, \lambda_d\}$. The \textbf{empirical spectral distribution} of $\HH$, called ${\mu}_{\HH,\;\alpha}$, is the probability measure
\begin{equation}\label{eq:wighted_spectral_density}
    \mu_{\HH} \defas \frac{1}{d}{\textstyle{\sum_{i=1}^d}} \delta_{\lambda_i},
\end{equation}
where $\delta_{\lambda_i}$ is the Dirac delta, a distribution equal to zero everywhere except at $\lambda_i$ and whose integral over the entire real line is equal to one.

Since $\HH$ is random, the empirical spectral distribution $\mu_\HH$ is a random measure (a random variable in the space of measures). Its expectation over $\HH$ is called the \textbf{expected spectral distribution} (e.s.d.) and we denote it
\begin{equation}
\mu \defas \EE_{\HH}[\mu_{\HH}]\,.
\end{equation}
\end{definition}

We can link the e.s.d. of $\HH$ to the convergence of a first order method on the distribution of $\HH$ In the following we'll consider $x_0-x^\star$ and $\HH$ to be independent, with $x_0-x^\star$ sampled isotropically. This isotropic hypothesis is not necessary and we could derive a similar analysis for  more general distribution  of $x_0-x^\star$.

\begin{restatable}{thm}{metrics} \label{thm: metrics}
Let $\xx_t$ be generated by a first-order method associated to the polynomial $P_t$,  $\mu$ the  e.s.d. of $H$ and $\mathbb{E}[(\xx_0-\xx^\star)(\xx_0-\xx^\star)^T]=R^2\textbf{I}$ Then we can write the convergence metrics at time step  $t$ as:
\begin{align}\label{eq:error_norm_x}
  \mathbb{E}[\|\xx_t-\xx^\star\|^2] &= { R^2} \int {P_t^2(\lambda) d\mu(\lambda)} \\
    \mathbb{E}[f(\xx_t)-f(\xx^\star)]&=R^2\int P_t^2(\lambda)\lambda d\mu(\lambda)\\
    \mathbb{E}[||\nabla f(\xx_t)||^2_2]&=R^2\int P_t^2(\lambda)\lambda^2d\mu(\lambda) 
\end{align}
\end{restatable}
This shows that the polynomials are an excellent abstraction: in the following we'll refer directly to the polynomials associated to a given method and omit the $R^2$ term associated to the initialization. We'll simply refer to metric $l$ as the metric associated to the added $\lambda^l$ term, i.e. the gradient norm is metric $l=2$\\
\paragraph{}
This framework is further linked to the field of \textbf{orthogonal polynomials} by the following proposition which gives a construction of an optimal method w.r.t. a given distribution 
\begin{restatable}{prop}{optimality}\cite{pedregosa2020acceleration}
 \label{prop: optimality}
 Let $P_t^\star$ be defined as
 \begin{equation}
     P_t^\star:={\arg \min}_{P_t(0)=1} \int P_t^2(\lambda) \lambda^l d\nu(\lambda)
 \end{equation}
 then $(P_t^\star)$ is the family of residual orthogonal polynomials w.r.t. to $\lambda^{a+1}d\nu$
\end{restatable}

This theorem further implies that the  optimal first-order method is a momentum method as Favard's theorem \cite{marcellan2001favard} tells us the residual orthogonal polynomials w.r.t. a given distribution can be related through a \textbf{three term recurrence}:

\begin{equation}
    P_{t+1}(\lambda)=a_tP_t(\lambda)+b_t\lambda P_t(\lambda)+(1-a_t)P_{t-1}(\lambda)
\end{equation}
Following remark \ref{rmk: momentum based}, the optimal method is derived from this recurrence as:
\begin{equation}
    \xx_{t+1}=\xx_t+(a_t-1)(\xx_t-\xx_{t-1})+b_t\nabla f(\xx_t)
\end{equation}



\section{Generalized Chebyshev and Laguerre methods}
Being able to write the rates in terms of the \textit{expected spectral distribution} ties the average case framework to the field of \textit{random matrix theory}. Indeed, because of results from this field, certain e.s.d's are considered more natural than others. We illustrate this and motivate following considerations with:
\begin{prop}[Marchenko Pastur Theorem]
Let $X_n$ be a $m\times n$ random matrix with $X_{ij}$ i.i.d with variance $\sigma^2$ and $Y_n=\frac{1}{n}X_nX_n^T$. Let $\mu_n$ be the expected spectrum of $Y_n$, then, as $n\rightarrow\infty$, $\frac{m}{n} \rightarrow r$:

\begin{align*}
&\mu_n\xrightarrow{\text{weakly}}{}\max(0,1-\frac{1}{r})\delta_0+\mu_{MP} \\
&d\mu_{MP}(\lambda)=\frac{1}{2\pi\sigma^2}\frac{\sqrt{(\lambda^+-\lambda)(\lambda-\lambda^-)}}{r\lambda}
\end{align*}
With $\lambda^+=\sigma^2(1+\sqrt{r})^2$, $\lambda^-=\max(0,\sigma^2(1-\sqrt{r})^2)$

\end{prop}
The Marchenko Pastur distribution $\mu_{MP}$ can be considered a natural first model for e.s.d's as it arises universally from i.i.d. noise in the matrix entries. Note there's no specific distribution of $X_{ij}$ considered. \\
When $r=1 d\mu_{MP}\propto \lambda^{-1/2}\sqrt{\lambda^+-\lambda}$, Though practical e.s.d's diverge from it, the concentration near $0$ is often verified.
\cite{pedregosa2020acceleration} first derived the optimal method wrt. $\mu_{MP}$, and \cite{paquette2020halting} derived Nesterov's rates under it. As we are mainly concerned with being robust, a natural step is to consider the Beta weights $d\mu(\lambda)\propto\lambda^\xi(L-\lambda)^\tau$ but we are mainly interested in distributions with similar concentrations near $0$, i.e. $\xi\approx -1/2$.\\
The optimal method w.r.t. $\mu$ and metric $l$ is associated to a shifted Jacobi polynomial $\tp_t^{\alpha,\beta}$ with $\beta=\xi+l+1, \alpha=\tau$. When $\alpha=\beta=-1/2$ we retrieve the \textit{Chebyshev Method} \cite{hestenes1952methods}, so we call this the Generalized Chebyshev Method. \\
We'll also consider the Laguerre method, which is optimal w.r.t. $d\mu(x)=\frac{x^\alpha e^{-x}}{\Gamma(\alpha+1)}$, taking $\alpha$ as a parameter. This method is proposed to optimize non-smooth functions\\
Both these methods are generalizations of one that have been proposed in \cite{pedregosa2020acceleration}. We derive the coefficients associated to these methods in appendix \ref{jacobi recurrence}.\\
\begin{remark}
The Generalized Chebyshev also takes the largest eigenvalue $L$ as a parameter, but the rates we'll show are robust to an \textit{overestimation} of $L$
\end{remark}

\section{Robust Average Case Rates}
We will establish our assumption over the spectral distributions. It effectively allows us to parametrize all of our distributions of interest in a way that characterize the asymptotic convergence, diving them into equivalence classes .

\begin{assumption}
We will write $\nu_{\tau,\xi}$ for a continuous distribution supported in $(0,L]$ s.t. $\nu_{\tau,\xi}'(x)>0$ for $x\in [0,L]$, $d\nu_{\tau,\xi}=\Theta( \lambda^\xi)$ near $0$ and $d\nu_{\tau,\xi}=\Theta( (L-\lambda)^\tau)$ near $L$. 
\label{assumption}
\end{assumption}

We argue this is a much more mild assumption to be made than the exact spectral distribution and it covers any distribution modeling a smooth convex problem. \\
The $\xi$ works as a measure of how close we are to the worst case scenario, as it approaches $-1$. Samples in finite dimension, of distributions with high values of $\xi$, will work as strongly convex functions in practice.
\paragraph{}
We show that $\nu_{\tau,\xi}$ indeed behaves like an equivalence class when considering the asymptotics of the convergence of a Jacobi method: only the concentrations near the edge matter. We do this by singling out from each of these classes the beta distributions for which we can compute the rates, then show the rates to be the same inside $\nu_{\tau,\xi}$.\\ 


\begin{restatable}{theorem}{robustjacobi}\label{thm: jacobirates}
A Generalized Chebyshev Method with parameters $(\alpha,\beta)$ applied to a problem with e.s.d. as in assumption \ref{assumption} has rates:

\begin{align}
\mathbb{E}[f(\xx_t)-f(\xx^\star)]&\sim L\cdot C^{\alpha,\beta}_{1,\nu}
    \left\{\begin{array}{ll}
    t^{-1-2\beta} &\mbox{if } 
		  \alpha<\tau+1/2 \text{ and } \beta <\xi+3/2\\
		  t^{-2(\xi+2)}\log t& \mbox{if } 
		  \alpha=\tau+1/2 \text{ and } \beta =\xi+3/2\\
		  t^{2(\max\{\alpha-\beta-\tau,-\xi-1\}-1)}& \mbox{if } 
		  \alpha>\tau+1/2 \text{ or } \beta >\xi+3/2
	\end{array}\right.\\
	\mathbb{E}[||\nabla f(\xx_t)||^2_2]&\sim L^2\cdot C^{\alpha,\beta}_{2,\nu}
        \left\{\begin{array}{ll}
    t^{-1-2\beta} &\mbox{if } 
		  \alpha<\tau+1/2 \text{ and } \beta <\xi+5/2\\
		  t^{-2(\xi+3)}\log t& \mbox{if } 
		  \alpha=\tau+1/2 \text{ and } \beta =\xi+5/2\\
		  t^{2(\max\{\alpha-\beta-\tau,-\xi-2\}-1)}& \mbox{if } 
		  \alpha>\tau+1/2 \text{ or } \beta >\xi+5/2
	\end{array}\right.
\end{align}
Where $C^{\alpha,\beta}_\nu$ is a distribution dependent constant.
\end{restatable}
\begin{figure}[H]
    \centering
    %\includegraphics[width=10 cm]{imgs/diagram.PNG}
    \includegraphics[width=4 cm]{imgs/colormap.png}
    
    \caption{Colormap for the function value rates  for the Marchenko Pastur distribution. The color represents the negative exponent, e.g. lower is better.}
    \label{fig:my_label2}
\end{figure}



Theorem \ref{thm: optimality} shows that a proper choice of $\alpha,\beta$ can indeed make the Jacobi polynomial asymptotically optimal w.r.t. to any $\nu_{\tau,\xi}$. 

\begin{restatable}{theorem}{jacoptimal}\label{thm: optimality}
Let $\nu$ follow assumption \ref{assumption}.
The optimal asymptotic rate for $\mathbb{E}[f(\xx_t)-f(\xx^\star)]$ is $t^{-2(\xi+2)}$ and is attained by the Chebyshev Method with parameters $(\tau,\xi+2)$. \\
The optimal rate for $\mathbb{E}[||\nabla f(\xx_t)||^2_2]$ is $t^{-2(\xi+3)}$ and is attained by the Chebyshev Method with parameters $(\tau,\xi+3)$.
\end{restatable}

For the function value ($l=1$), we find rates that approach $t^{-2}$ as $\xi\rightarrow -1$, showing the worst case as a limit (over the considered distribution) on the average case.
\begin{remark}
We can contextualize our results on the field of orthogonal polynomials as asymptotics on the values of the \textit{Christoffel Functions} at $0$ \cite{totik2005orthogonal}:

\begin{equation}
    \lambda_t(\mu,x)=\inf_{P_t(x):=1,\deg(P_t)\leq t}\int P_t^2 d\mu=\Big( \sum_{k=0}^tp_k(x;\mu)^2\Big)^{-1}
\end{equation}
 For the equivalence classes $\nu_{\tau,\xi}$
\end{remark}
We remark that the above theorems imply that, at least asymptotically, the Jacobi method is robust for a suboptimal choice of parameters up to $1/2$ below the optimal choice of $\beta$ and infinitely above. \\
For completeness, we also derive worst case rates for the Jacobi method:

\begin{restatable}{prop}{worstcase}
Let $f$ be a convex, L-smooth quadratic function. Then, For the Generalized Chebyshev Method with parameters $(\alpha,\beta)$ we have:

\begin{align}
 f(\xx_t)-f(\xx^\star) \leq C_1L\left\{
    \begin{array}{cc}
           t^{2(\alpha-\beta)} &\text{if} \hspace{0.5 cm} \alpha>\beta-1 \\
         t^{-1-2\beta}, \hspace{1 cm} &\text{if} \hspace{0.5 cm} \alpha\leq \beta-1\hspace{0.5 cm} \beta\leq \frac{1}{2}\\
         t^{-2}, \hspace{1 cm} &\text{if} \hspace{0.5 cm} \alpha\leq\beta-1\hspace{0.5 cm} \beta\geq \frac{1}{2} 
    \end{array}
    \right .\\
    ||\nabla f(\xx_t)-f(\xx^\star)||\leq C_2L^2\left\{
    \begin{array}{cc}
           t^{2(\alpha-\beta)} &\text{if} \hspace{0.5 cm} \alpha>\beta-2 \\
         t^{-1-2\beta}, \hspace{1 cm} &\text{if} \hspace{0.5 cm} \alpha\leq \beta-2\hspace{0.5 cm} \beta\leq 3/2\\
         t^{-4}, \hspace{1 cm} &\text{if} \hspace{0.5 cm} \alpha\leq\beta-2\hspace{0.5 cm} \beta\geq 3/2
    \end{array}
    \right . 
\end{align}
\end{restatable}


For the function value ($l=1$) and a reasonable choice of $\alpha,\beta$ this means effectively that the worst case rates are $t^{-2}$ 

\paragraph{}
\cite{nesterov2003introductory} has shown that the Nesterov matches up to a a constant factor a lower bound on the worst case complexity of non strongly convex problems. A natural question is if this performance would translate to good average case rates. \\
We'll extend \cite{paquette2020halting} proof for the Nesterov method under the MP distribution  
\begin{restatable}{theorem}{nesterovrates}
Let $\nu$ as in assumption \ref{assumption}, then, for the Nesterov method:
\begin{align}
    \mathbb{E}[f(\xx_t)-f(\xx^\star)]&\sim C'_{1,\nu}
    \Big\{\begin{array}{ll}
		  t^{-2(\xi+2)}& \mbox{if } 
		  \xi<-1/2\\
		  t^{-3}\log t& \mbox{if } 
		  \xi=-1/2\\
		  t^{-(\xi+7/2)}& \mbox{if } 
		  \xi>-1/2
	\end{array}\\
	\mathbb{E}[||\nabla f(\xx_t)||^2_2] &\sim C'_{2,\nu}
		  t^{-(\xi+9/2)}
\end{align}
\end{restatable}

The optimality gap of nesterov is $t{\xi+l-1/2}$, when $\xi+l>1/2$, $\log t$ when $\xi+l=1/2$ and $0$ otherwise.
This shows that Nesterov is almost optimal when the concentrations near $0$ are relatively high, i.e. low $\xi$\\

\begin{restatable}{theorem}{gdrates}
Let $\nu$ as in assumption \ref{assumption}, then for gradient descent:
\begin{align}
\mathbb{E}[f(\xx_t)-f(\xx^\star)]=\Theta(t^{-(\xi+2)})\\
	\mathbb{E}[||\nabla f(\xx_t)||^2_2] =\Theta(t^{-(\xi+3)})
\end{align}

\end{restatable}

Observe, for the function value, that we  find the $t^{-2}$ rates for Nesterov and $t^{-1}$ for Gradient Descent  when $\xi\rightarrow-1$

\begin{table}[H]
    \centering
    \begin{tabular}{c|c|c|c|c}
         $(\tau,\xi)$/Method& Chebyshev ($\frac{1}{2},\frac{5}{2}$) & Chebyshev ($\frac{1}{2},\frac{3}{2}$) &  Nesterov  & G.D. \\
         \hline
         ($\frac{1}{2},\frac{1}{2}$)&$t^{-5}$ & $t^{-4}$ & $t^{-4}$ & $t^{-\frac{5}{2}}$\\
         \hline
         ($\frac{1}{2},\frac{-1}{2}$)&$t^{-3}$ & $t^{-3}$ & $t^{-3}\log t$ & t^{-\frac{3}{2}}
    \end{tabular}
    \caption{Comparison of asymptotic rates for the function-value for different methods an $(\tau,\xi)$ values}
    \label{tab:theoretic rates}
\end{table}

Lastly, we consider the optimal rates for a Gamma distribution.

\begin{restatable}{theorem}{laguerrerates}
Let $\alpha>-1$ and $\mu_\alpha$ be a Gamma distribution, i.e. $d\mu_\alpha(x)=\frac{x^\alpha e^{-x}}{\Gamma(\alpha+1)}$. The optimal rates are given by the Laguerre method of appropriate tuning and:
\begin{equation}
    \mathbb{E}[f(\xx_t)-f(\xx^\star)]=\Theta(t^{-(\alpha+2)})
    %%we can get the constants here
\end{equation}
\end{restatable}
Note that this result does not have the same universality of the others because of the non-compacity of the distribution's support.\\
These rates are contrasted  to the worst case lower bound on the optimization of non-smooth functions by first order methods, which gives:
\begin{equation*}
    f(\xx_k)-f(\xx^\star)\geq\frac{C}{\sqrt{t}}
\end{equation*}
These rates are not found when $\alpha\rightarrow-1$, indicating that the worst case is specially pessimistic in this scenario.
\section{Experiments}
We estimated empirically the asymptotic rates from table \ref{tab:theoretic rates} by doing linear regression on the log-log domain. \\
We synthesize the e.s.d's  either by means of the Marchenko Pastur that enables us to simulate $(\tau,\xi)$ values of $(1/2,3/2)$ and $(1/2,5/2)$. Other values are simulated by sampling from the corresponding Beta distribution and applying a random rotation to the diagonal matrix comprising those samples.\\ 
As ours are the asymptotic rates on the infinite dimensional case, i.e. we take $d\rightarrow \infty$ and then $t \rightarrow \infty$, they are representative of the experiments on the regime $t<d$.

\begin{figure}[H]
    \centering
    \includegraphics[width=5 cm]{imgs/mp/log f.png}\includegraphics[width= 5 cm]{imgs/mp/log grad.png}
    
    
    \caption{Rates for a synthetic problem, simulating the Marchenko Pastur distribution. \textit{Left:} function value. \textit{Right}: gradient norm}
    \label{fig:my_label}
\end{figure}


\begin{figure}[H]
    \centering
    \includegraphics[width=5 cm]{imgs/inception/spectrum.png}\includegraphics[width= 5 cm]{imgs/mnist/spectrum.png}
    \includegraphics[width=5 cm]{imgs/inception/log grad.png}\includegraphics[width= 5 cm]{imgs/mnist/grad log.png}
    
    
    \caption{Spectrum and gradient norm rates for regression problems. \textit{Left:} CIFAR-10 Inception features \textit{Right}: MNIST features}
    \label{fig:my_label}
\end{figure}


\begin{figure}[H]
    \centering
    \includegraphics[width=5 cm]{imgs/theo vs practic.png}\includegraphics[width= 5 cm]{imgs/slopes.png}
    
    
    \caption{Left: Empirical vs theoretical rates for fixed $\xi$ and varying $\beta$. Right: Empirical slopes for varying $\xi$}
    \label{fig:my_label}
\end{figure}

\begin{table}[H]
    \centering
    \begin{tabular}{c|c|c|c}
         $(\tau,\xi)$/Method& Chebyshev ($\frac{1}{2},\frac{5}{2}$) & Chebyshev ($\frac{1}{2},\frac{7}{2}$) &  Nesterov  \\
         \hline
         ($\frac{3}{2},\frac{1}{2}$)&$t^{-3.96}$? & $t^{-4.93}$ & $t^{-4.01}$\\
         \hline
         ($\frac{1}{2},\frac{1}{2}$)&$t^{-3.00}$ & $t^{-2.83}$ & $t^{-2.68}$
         
    \end{tabular}
    \caption{Empirical slopes for the function-value }
    \label{tab:experimental rates}
\end{table}
%%Chebyshev \beta=5/2 is not giving the expected gradient performance ...


\section{Conclusion}
\printbibliography
\appendix
\newpage
\section{Proofs of Section 2}
\metrics*
\begin{proof}
\newcommand\xinit{\xx_0-\xx^\star}
We remark that by the definition of the expected spectral distribution $\mu$ of $\HH$, we have for continuous $g$:
\begin{equation}
    \EE_H[g(tr(\HH))]=\int g(\lambda)\dif mu(\lambda)
\end{equation}
We know that $\xx_t-\xx^\star=P_t(\HH)(\xinit)$. We can write $||\xx_t-\xx^\star||^2$ in terms of a trace and use the independence of $\HH$ and $\xinit$ to connect it to the e.s.d.:
\begin{align}
    \mathbb{E}||\xx_t-\xx^\star||^2&=\EE[ tr((\xinit)^TP_t(\HH)^2(\xinit))] \\
    &=\EE_{\HH,\xinit} [tr(P_t(\HH)^2(\xinit)(\xinit)^T]\\
    &=\EE_\HH\left[P_t(\HH)^2\EE_{\xinit}[(\xinit)(\xinit)^T])\right]  \\
    &=R^2\EE_\HH[P_t(tr(\HH))^2]=R^2 \int P_t(\lambda)^2\dif\mu(\lambda)
\end{align}
For the gradient and function value the reasoning is the same by noticing:
\begin{align}
    \EE[f(\xx_t)-f(\xx^\star)]&=\EE[ tr((\xinit)^TP_t(\HH)\HH P_t(\HH)(\xinit))]\\
    &=\EE_\HH[(\lambda P_t)(\tr(H))^2] 
\end{align}
Where $\lambda P_t$ is also a  polynomial. As $\nabla f(\xx_t)=\HH(\xx_t-\xx^\star)$
\begin{align}
    \EE||\nabla f(\xx_t))||^2&=\EE[ tr((\xinit)^TP_t(\HH)\HH^2 P_t(\HH)(\xinit))]\\
    &=\EE_\HH[(\lambda^2 P_t)(\tr(H))^2]
\end{align}


\end{proof}

\optimality*
\begin{proof}
We differentiate the expression for the metrics w.r.t. to the coefficients of the polynomials:
\begin{align*}
    \frac{d}{da_k}\left(\int \lambda^lP_t^2(\lambda)d\mu(\lambda)\right)&=\int\lambda^l\frac{\dif}{\dif a_k}\left(\sum_{k=0}^t a_k\lambda^kP_t(\lambda) \right) \dif \mu(\lambda)=
    \\&=2\cdot\left(\int \lambda^{l+k}P_t(\lambda)d\mu(\lambda)\right)=0
\end{align*}
This means that $P_t(\lambda)$ is orthogonal to any polynomial of degree $t-1$  w.r.t to the intern product $\langle.,.\rangle_{\lambda^{l+1}d\mu}$

\end{proof}

\section{Coefficients  of the Jacobi Method}
We'll first state two lemmas that allow us to construct the optimal polynomials. With them in hand the procedure is trivial.

\begin{lemma}
Let $(\tp_t)$ be a family polynomials  following  
\begin{equation*}
    \tp_t(\lambda)=(\alpha_t+\beta_t\lambda)\tp_{t-1})\lambda)+\gamma_t\tp_{t-2}(\lambda)
\end{equation*}
With $\tp_0$ a constant polynomial and $\tp_t\neq0, \forall t$. Then:
\begin{equation}
    P_t(\lambda)=(a_t+b_t\lambda)P_{t-1}(\lambda)+(1-a_t)P_{t-2}(\lambda)
\end{equation} Is the recurrence for $P_t(\lambda)=\tp_t(\lambda)/\tp_t(0)$. With:
\begin{align}
    a_t&=\delta_t\alpha_t \\
    b_t&=\delta_t\beta_t \\
    \delta_t&=(\alpha_t+\gamma_t\delta_{t-1}) \hspace{0.5 cm } (\delta_0=0)
\end{align}
\end{lemma}
The proof of this is presented in \cite{pedregosa2020acceleration}. Further, we know how to compute the recurrence for the polynomials of a shifted distribution:
\begin{lemma}

Let $(\tp_t)$ be a family polynomials orthogonal w.r.t  following  
\begin{equation}    \label{rec}
    \tp_t(\lambda)=(\alpha_t+\beta_t\lambda)\tp_{t-1}(\lambda)+\gamma_t\tp_{t-2}(\lambda
\end{equation}
And define polynomials $P_t$ s.t. :
$$
P_t(m(\lambda))=\tp_t(\lambda)
$$
With $m(\lambda)=a\lambda+b$ a non singular affine transform. Then $P_t$ follows a recurrence like in eq. \eqref{rec}, with:
\begin{align}
    \alpha_t'&=\alpha_t+b\beta_t \\
    \beta'_t&=a\beta_t\\
    \gamma'_t&=\gamma_t
\end{align}
\end{lemma}
The lemma is self-evident by considering eq. \eqref{rec} with argument $m^{-1}(\lambda)$
\paragraph{}
\label{jacobi recurrence}
Then to get  the recurrence relation for the residual polynomial w.r.t $x^\beta(L-x)^\alpha$, we begin by the standard jacobi polynomials, which are orthogonal w.r.t $(1-x)^\alpha(1+x)^\beta$ and follow a recurrence according to $\alpha_t,\beta_t,\gamma_t$ below, shift the distribution according to $\eta(x)\ldots$, and then transform to the residual ones:
\begin{prop}
The residual polynomials w.r.t. $d\mu(\lambda)=\lambda^\beta(L-\lambda)^\alpha$, follow the recurrence:
\begin{equation*}
    P_t(\lambda)=(a_t+b_t\lambda)P_{t-1}(\lambda)+(1-a_t)P_{t-2}(\lambda)
\end{equation*}
With:
\begin{align}
    \alpha_t&=\frac{(\alpha^2-\beta^2)(2n+\alpha+\beta+1)}{
            2(n+1)(n+\alpha+\beta+1)(2n+\alpha+\beta)} \\
        \beta_t&=\frac{(2n+\alpha+\beta+1)(2n+\alpha+\beta+2)}{
            (2(n+1)(n+\alpha+\beta+1)} \\
        \gamma_t&=-\frac{(n+\alpha)(n+\beta)(2n+\alpha+\beta+2)}{
            (n+1)(n+\alpha+\beta+1)(2n+\alpha+\beta)} \\
        \tilde{a}_t&=\alpha_t-\beta_t \\
        \tilde{b}_t&=\frac{2}{L}\beta_t \\
        \delta_t&=\frac{1}{\tilde{a}_t+c_t\delta_{t-1}}  \hspace{1 cm} (\delta_0=0)\\
        a_t,b_t&=\delta_t\tilde{a}_t,\delta_t\tilde{b}_t
\end{align}


\end{prop}

\section{Proofs of section 3}
In the following we'll consider shifted versions of the spectral distributions. This shift is written as an affine transform $m(\lambda):[0,L]\rightarrow [-1,1]$ because most results in the theory of orthogonal polynomials are stated in terms of distributions supported in $[-1,1]$. \\
This can be seen as an additional layer of abstraction because the quantities evaluated with the shifted distributions and polynomials are proportional, i.e. if $P_t(x)=\tp_t(m(x))$ and $\mu'(x)=\tmu'(m(x))$:
\begin{equation}
    \int P_t^2(x)\mu'(x)\dif x\propto \int \tp_t^2(x)\tmu'(x)\dif x
\end{equation}
So all the asymptotics are the same and we consider $\nu$ restricted to $[-1,1]$. 
The Jacobi polynomials $\JP_t$ are orthogonal w.r.t; $d\mu(x)=(1-x)^\alpha(1+x)^\beta$, we note that most results give them in terms of normalization $\tp_t^{\alpha,\beta}(-1)=(-1)^t\binom{t+\beta}{t}$. We'l write $\tp^{\alpha,\beta}_t$ for this normalization and $\JP_t$ for the residual polynomials
\robustjacobi*
\begin{proof}
We'll prove that for any $\alpha$ and $\beta$, $\xi,\tau>-1$, $l>0$ and $\nu$ following Assumption \ref{assumption}, we have:
\begin{equation*}
    \int P_t^{\alpha,\beta}(x)^2x^l d\nu_{\tau,\xi-l}(x) \sim  L^lC^{\alpha,\beta}_\nu\left\{
	\begin{array}{ll}
		  t^{-1-2\beta}& \mbox{if } 
		  \alpha<\tau+1/2 \text{ and } \beta <\xi+1/2\\
		  t^{-2(\xi+1)}\log t& \mbox{if } 
		  \alpha=\tau+1/2 \text{ and } \beta =\xi+1/2\\
		  t^{2(\max\{\alpha-\beta-\tau,-\xi\}-1)}& \mbox{if } 
		  \alpha>\tau+1/2 \text{ or } \beta >\xi+1/2
	\end{array}
\right.
\end{equation*}
We first state a lemma shown in \cite{van1995weak} relating to the weak convergence of the orthogonal polynomials:

\begin{lemma}
 Let $\mu$ be a measure and $(p_t)$ it`s family of orthonormal polynomials s.t.  $p_t$ follow the recurrence:
 \begin{equation*}
     xp_t(x)=a_tp_{t+1}(x)+b_tp_t(x)+a_{t-1}p_{t-1}(x)
 \end{equation*}
 and $a_t,b_t$ converge respectively to $a,b$. Then for any $f$  continuous and bounded:
\begin{equation}
    \int f(x)p_t^2(x)d\mu(x) \rightarrow \frac{1}{\pi}\int_{-1}^1 \frac{f(x)}{\sqrt{1-x^2}}dx   
\end{equation}
\label{wk}
\end{lemma} The normalization of $\tp^{\alpha,\beta} _t$ is s.t. \cite{szego1975orthogonal} (4.3.3):
\begin{equation}
    \int_{-1}^1\tp_t^{\alpha,\beta}(x)(1-x)^\alpha(1+x)^\beta \dif x=\frac{2^{\alpha+\beta+1}}{2n+\alpha+\beta+1}\frac{\Gamma(n+\alpha+1)\Gamma(n+\beta+1)}{\Gamma(n+1)\Gamma(n+\alpha+\beta+1)}=\Theta(t^{-1})
\end{equation}

The residual polynomials then are s.t. $|P^{\alpha,\beta}_t|=\Theta(t^{-\beta})\tp^{\alpha,\beta}_t$.\\
We also state the following result (Exercise 91, Generalisation of 7.34.1) from \cite{szego1975orthogonal}:
\begin{lemma}
We have:
\begin{align}
    &\int_0^1(1-x)^\tau P_t^{\alpha,\beta}(x)^2dx \sim\Theta( h_{\tau}^\alpha) \\
    &h_{\tau}^\alpha:=
\left\{
	\begin{array}{ll}
		t^{2(\alpha-\tau-1)}  & \mbox{if } \alpha>\tau+1/2 \\
		t^{-1}\log t   & \mbox{if } \alpha=\tau+1/2 \\
		t^{-1}   & \mbox{if } \alpha<\tau+1/2
	\end{array}
\right.
\end{align}
 \label{jacobi lemma}
\end{lemma}
Noting that $\tp^{\alpha,\beta}_t(x)=(-1)^t\tp_t^{\beta,\alpha}(-x)$, we can write:
\begin{equation}
    \int_{-1}^1\tp_t(x)^2(1-x)^\tau(1+x)^\xi dx = \Theta\left(\int_0^1(1-x)^\tau|\tp_t^{\alpha,\beta}(x)|^2dx\right) +\Theta\left(\int_0^1(1-x)^\xi|\tp_t^{\beta,\alpha}(x)|^2dx\right) \label{int decomp}
\end{equation}
We can then show our result for $\dif\nu_{\tau,\xi-l}(x)=x^{\xi-l}(L-x)^\alpha$ by carefully considering each of the cases on Lemma~\ref{jacobi lemma} and the maximum of each term in eq. \ref{int decomp}, and an added $t^{-2\beta}$ from the different normalization. \\
It rests to show that all cases can be brought down to this one. Indeed we show simply:
\begin{equation}
    \int_0^1 \tp_t^{\alpha,\beta}(x)^2\dif\nu_{\tau,\xi}(x)= \Theta\left(\int_0^1(1-x)^\tau\tp_t^{\alpha,\beta}(x)^2dx \right)
\end{equation}
And the rest follows from the same arguments. Let $\epsilon$ s.t.
\begin{equation}
    x\geq 1-\epsilon \Rightarrow |\dif\nu_{\tau,\xi}-A(1-x)^\tau|\leq  B(1-x)^\tau \label{eq: epsilon}
\end{equation}
We observe that for $0<x<1-\epsilon$, $f(x)=\frac{d\nu_{\tau,\xi}}{(1-x)^\alpha(1+x)^\beta}$ is bounded. \\
We get from an application of \ref{wk}, and the observation that $\tp_t^{\alpha,\beta}=\mathcal{N}_tp_t^{\alpha,\beta}$, with $\mathcal{N}_t=\Theta(t^{-1/2})$:
\begin{align}
    \underbrace{\int_0^1(1-x)^\tau\tp_t^{\alpha,\beta}(x)^2\dif x}_{\Theta( h_\tau^\alpha)}&=\underbrace{\int_0^{1-\epsilon}(1-x)^\tau\tp_t^{\alpha,\beta}(x)^2\dif x}_{\Theta( t^{-1})
    } +\int_{1-\epsilon}^1(1-x)^\tau\tp_t^{\alpha,\beta}(x)^2\dif x \Rightarrow\\
    &\int_{1-\epsilon}^1(1-x)^\tau\tp_t^{\alpha,\beta}(x)^2\dif x  =\Theta( h_\tau^\alpha) \\
    \int_0^1 \tp_t^{\alpha,\beta}(x)^2\dif\nu_{\tau,\xi}(x)&=
    \underbrace{\int_0^{1-\epsilon} \tp_t^{\alpha,\beta}(x)^2 f(x) (1-x)^\alpha(1+x)^\beta\dif x)}_{\Theta( t^{-1})
    }+\Theta\left(
    \underbrace{\int_{1-\epsilon}^1(1-x)^\tau\tp_t^{\alpha,\beta}(x)^2\dif x}_{\Theta( h_\tau^\alpha)}\right)
\end{align}


\end{proof}

\jacoptimal*
\begin{proof}
We'll prove that for $\tau,\xi>-1$ If $\alpha = \tau$ and $\beta = \xi+l+1$ (i.e., are optimal), the rate of convergence reads
\begin{equation}
    \min_{P_t(0)=1}\int P_t^2(\lambda)\lambda^ld\nu(\lambda)=\Theta\left( \int_{0}^l  \tp_t^{\alpha,\beta}(\lambda)^2(L-\lambda)^\tau\lambda^{\xi+l}\dif \lambda\right) =\Theta( t^{-2(\xi+l+1)})
\end{equation}
Showing the second equality is easy by considering theorem \ref{thm: jacobirates}, and that is further the minimum asymptotic rate for the Beta distribution. \\
As $\tp^{\alpha,\beta}_t$ has the same rate on $\nu$ as  on the Beta distribution , the minimum rate for $\nu$ is lower bounded by the r.h.s. \\
We argue that, setting $P^\nu_t=\frac{p^\nu}{p^\nu_t(0)}$ the optimal residual and orthonormal polynomials and $\mu_{\tau,\xi}$ the Beta distribution  w.r.t, $P^\nu_t$ must have the same rate on $\nu$ as it does on $\nu$. Indeed, setting $\epsilon_1,\epsilon_2$ as in eq. \ref{eq: epsilon}, we argue:
\begin{align}
    \int_{1-\epsilon_2}^{1}P_t^\nu(x)^2d\nu(x)&=\left(\Theta\int_{1-\epsilon_2}^{1}P_t^\nu(x)^2d\mu(x)\right)\\
    \int_{-1}^{-1+\epsilon_1}P_t^\nu(x)^2d\nu(x)&=\Theta\left(\int_{-1}^{-1+\epsilon_1}P_t^\nu(x)^2d\mu(x)\right)\\
    \int_{-1+\epsilon_1}^{1-\epsilon_2}P_t^\nu(x)^2d\nu(x)&=\Theta\left(\int_{-1+\epsilon_1}^{1-\epsilon_2}P_t^\nu(x)^2d\mu(x)\right)=\Theta\left(\frac{1}{p_t^\nu(-1)^2}\right)\\
\end{align}
Where the first two equations come from the fact that $\nu=\Theta(\mu)$ near $-1$ and $1$ and the third from lemma \ref{wk}.\\
This effectively upper bounds the rates on $\nu$ because the rates of $P_t^\nu$ on $\mu_{\tau,\xi}$ can't be lower than $-2(\xi+1)$.
\end{proof}




\worstcase *
\begin{proof}
rates]
We'll prove that:  $\sup_{x\in[0,L]}x^lP_t^{\alpha,\beta}(x)^2=O(L^lt^{v(\alpha,\beta,l)})$. Where:
\begin{equation}
    v(\alpha,\beta,l)=\left\{
    \begin{array}{cc}
           2(\alpha-\beta) &\text{if} \hspace{0.5 cm} \alpha>\beta-l \\
         -1-2\beta, \hspace{1 cm} &\text{if} \hspace{0.5 cm} \alpha\leq \beta-l\hspace{0.5 cm} \beta\leq l-\frac{1}{2}\\
         -2l, \hspace{1 cm} &\text{if} \hspace{0.5 cm} \alpha\leq\beta-l\hspace{0.5 cm} \beta\geq l-\frac{1}{2} 
    \end{array}
    \right . 
\end{equation}
From \cite{szego1975orthogonal}, Theorem 7.32.2, if $\theta<\frac{\pi}{2}$:
\begin{equation}
    \tp_t^{\alpha,\beta}(\cos \theta)=\left\{ 
    \begin{array}{cc}
         O(t^{-1/2})  \hspace{1 cm} &\text{if} \hspace{0.5 cm} \alpha < -\frac{1}{2} \\
         O(t^{\alpha})  \hspace{1 cm} &\text{if} \hspace{0.5 cm} \alpha \geq -\frac{1}{2} , 0\leq\theta\leq ct^{-1} \\
         \theta^{-
         \alpha-1/2}O(t^{-1/2})  \hspace{1 cm} &\text{if} \hspace{0.5 cm} \alpha \geq -\frac{1}{2} , \theta> ct^{-1}
    \end{array}
    \right .
\label{lemma: worst case}
\end{equation}
We observe that, from the symmetry of the jacobi polynomials:
\begin{equation}
    \sup_{x\in[0,L]}x^lP_t^{\alpha,\beta}(x)^2 =\Theta\left( \max\left\{\sup_{x\in[0,1]}P_t^{\alpha,\beta}(x)^2,\sup_{x\in[0,1]}(1-x)^lP_t^{\beta,\alpha}(x)^2\right\}\right)
\end{equation}
The $(1-x)^l$ term,  corresponds to $(2\sin(\frac{\theta}{2}))^l$ in the variable $\theta$, which is $O(\theta^{2l})$. The rest follows from carefully considering the expressions given by eq. \ref{lemma: worst case}.
\end{proof}


\nesterovrates * 


\begin{proof}
We'll prove:
\begin{equation}
    \int_0^1P_t^{\text{Nes}}(\lambda)^2\lambda^l\dif\nu_{\tau,\xi-l}\sim C'_\nu
    \Big\{\begin{array}{ll}
          t^{-2(\xi+1)}& \mbox{if } 
		  0<\xi<1/2  \\
		  t^{-3}\log t& \mbox{if } 
		  \xi=1/2\\
		  t^{-(\xi+5/2)}& \mbox{if } 
		  \xi>1/2
	\end{array}
\end{equation}
\cite{paquette2020halting} has shown that the nesterov polynomials $P_t$ are asymptotically, in $t$:
\begin{equation}
    P_t(\lambda)\sim\frac{2J_1(t\sqrt{\alpha\lambda})}{t\sqrt{\alpha\lambda}}e^{-\alpha\lambda t/2}
\end{equation}
In the sense that:
\begin{equation}
    \int_0^1u^{l}\left[\Tilde{P_t^2(u)}-\frac{4J_1^2(t\sqrt{u})}{t^2u}e^{-u t}\right]4\dif\mu_{MP}(u)=O(t^{-(l+25/12))}
\end{equation}
The arguments can be easily used to show that such an integral is $O(t^{ -(\alpha+l+31/12)})$ when evaluated wrt a general $\dif\mu$ s.t $\mu'=\Theta(\lambda^\alpha)$ near $0$. \\
We can thus consider our integral  of interest substituting $P_t^\text{Nes}$ by it's Bessel asymptotic and dividing it into three regions, i.e. $[0,1]=[0,\frac{\epsilon}{t}]\cup[\frac{\epsilon}{t},\frac{\epsilon}{\sqrt{t}}]\cup[\frac{\epsilon}{\sqrt{t}},1]$ corresponding to two different regimes for the Bessel function. The first region will give us the asymptotic and the others we'll bound.\\
We consider first, for some $\epsilon>0$:
\begin{equation}
    \int_{\frac{\epsilon}{t}}^{\frac{\epsilon}{\sqrt{t}}} u^{\xi}\frac{4J_1^2(t\sqrt{u})}{t^2u}e^{-u t}\dif u
\end{equation}
We note the asymptotic for $J_1^2$:
\begin{equation}
    J_1^2(\sqrt{tv}) \sim \frac{1}{\pi\sqrt{tv}}(1+\cos(2\sqrt{tv}+2\gamma))
\end{equation}
Doing the change of variable $v=tu$, and identifying the upper limit of the interval, which is $\epsilon t^{1/2}$ to $\infty$:

\begin{align}
    \int_{\frac{\epsilon}{t}}^{\frac{\epsilon}{\sqrt{t}}} u^{\xi}\frac{4J_1^2(t\sqrt{u})}{t^2u}e^{-u t}\dif u &=\Theta\left(
    t^{-2-\xi}\int_\epsilon^\infty v^{\xi-1}J_1^2(\sqrt{tv})e^{-v}\dif v\right)\\
    &=\Theta\left( t^{-2-\xi}\int_\epsilon^\infty v^{\xi-1}\frac{1}{\pi\sqrt{tv}}e^{-v}\dif v \right)\\
    &=\Theta\left(t^{-\frac{5}{2}-\xi}\underbrace{\int_\epsilon^\infty v^{\xi-\frac{3}{2}}\frac{1}{\pi\sqrt{tv}}e^{-v}\dif v}_{\Gamma(\xi-\frac{1}{2},\epsilon) }\right)
\end{align}
Where the cosinus term goes to $0$ from the Riemann-Lebesgue lemma and $\Gamma$ is the incomplete Gamma function.\\
The term corresponding to the interval $[\epsilon t^{-1/2},1]$ is exponentially small. Indeed, because of the exponential $e^{-ut}$ it is  $O(e^{-\epsilon\sqrt{t}})$. This shows that the integral concentrates in a region that is closer and closer to $0$ and that only the behaviour of the distribution near $0$ matters.\\
We have for the $[0,\frac{\epsilon}{t}]$ region, doing the change of variables $v=t^2u$:
\begin{equation}
    \int_0^{\frac{\epsilon}{t}} u^{\xi}\frac{4J_1^2(t\sqrt{u})}{t^2u}e^{-u t}\dif u =\Theta\left(
    t^{-2(\xi+1)}\int_0^{t\epsilon} v^{\xi}\frac{J_1^2(\sqrt{v})}{v}e^{-\frac{v}{t}}\dif v\right)
\end{equation}
And the $e^{\frac{-v}{t}}$ is $\Theta(1)$. We have the following Bessel asymptotics:
\begin{align}
    \frac{J_1^2(\sqrt{v})}{v}&\sim \frac{1}{4}, \hspace{2 cm} v\rightarrow 0 \\
    \frac{J_1^2(\sqrt{v})}{v}&= O(v^{-3/2}), \hspace{1.0 cm} v\rightarrow \infty
\end{align}
So we divide this integral aswell:

\begin{align}
    t^{-2(\xi+1)}\int_1^{t\epsilon} v^{\xi}\frac{J_1^2(\sqrt{v})}{v}e^{-\frac{v}{t}}\dif v
    &=\Theta\left(t^{-2(\xi+1)}\int_\epsilon^{t\epsilon} v^{\xi-\frac{3}{2}}\dif v\right) =\Theta\left( I_\xi(t)t^{-\xi-\frac{5}{2}}\right)\\
    t^{-2(\xi+1)}\int_0^{1} v^{\xi}\frac{J_1^2(\sqrt{v})}{v}e^{-\frac{v}{t}}\dif v
    &=\Theta\left(t^{-2(\xi+1)}\int_0\epsilon^{1} v^{\xi} \dif v\right) =\Theta\left( t^{-2(\xi+1)}\right)
\end{align}
Where $I_\xi(t)=\log t$ if $\xi=\frac{1}{2}$ and $1$ otherwise. \\
The nesterov rate is then $I_\xi(t)t^{-\xi-\frac{5}{2}}$ if $\xi\geq\frac{1}{2}$ and $t^{-2(\xi+1)}$ if $0<\xi<\frac{1}{2}$
\end{proof}
\gdrates*
\begin{proof}
Considering that $P_t^\text{GD}(\lambda)=(1-\frac{\lambda}{L})^t$ we'll prove :
\begin{equation}
    \int_0^1(1-\lambda)^{2t}\lambda^l\dif\nu_{\tau,\xi-l}=\Theta(t^{-(\xi+l+1)}
\end{equation}
We know, for the Beta weights, that:
\begin{equation}
    \int_0^1(1-\lambda)^{2t+\tau}\lambda^{\xi+l}\dif\lambda=\frac{\Gamma(l+\xi+1)\Gamma(2t+\tau+1)}{\Gamma(et+l+\xi+\tau+2}=\Theta(t^{-(\xi+l+1)})
\end{equation}
We can indentify this asymptotic to the interval $\int_0^\epsilon$ for any $\epsilon$ because:
\begin{equation}
    \int_\epsilon^1(1-\lambda)^{2t+\tau}\lambda^{\xi+l}\dif\lambda=\mathcal{O}((1-\epsilon)^{2t})
\end{equation}
Then:
\begin{align}
    \int_\epsilon^1(1-\lambda)^{2t}\lambda^l\dif\nu_{\tau,\xi-l}&=\mathcal{O}((1-\epsilon)^{2t}) \\
    \int0^\epsilon(1-\lambda)^{2t}\lambda^l\dif\nu_{\tau,\xi-l}&=\Theta\left(\int_0^\epsilon(1-\lambda)^{2t+\tau}\lambda^{\xi+l}\dif\lambda\right)=\Theta(t^{-(\xi+l+1)})
\end{align}

\end{proof}

%%What happens when L is not properly set?
%% nesterov rates with tau< 1/2
\end{document}